\documentclass[12pt]{article}

\usepackage[margin=1in]{geometry}
\usepackage[table,xcdraw]{xcolor}
\usepackage[utf8]{inputenc}
\usepackage{amsmath,amsthm,amssymb}
\usepackage{graphicx}

\def\code#1{\texttt{#1}}

\begin{document}

\title{PCP - Trabalho 2}
\author{Matheus Ambrozio \& Renan Almeida}

\maketitle

O trabalho foi desenvolvido conforme a especificação. A três diferentes versões do código podem ser encontradas na pasta \code{/src} em \code{/problem-01}, \code{/problem-02/version-01}, e \code{/problem-02/version-02}.

A função usada para teste está definida em \code{/src/weird.c}. Utilizamos a função weird1 para os testes. Calculamos a área usando o intervalo [-3,~3]. A função matemática correspondentes é:

\begin{align}
& f1(x) = (x - 1) * (x + 3)^2 * (x - 2)^3 * (x + 1)^4
\end{align}

\section*{Medições}

Realizamos os testes nos nós n00-n07, cada nó com 4 núcleos e possibilidade de conter no máximo 4 processos. As medições são geradas através da execução do script \code{tester.lua}.

As medições foram feitas usando a função \code{MPI\_Wtime}. Abaixo, temos a tabela com os resultados em segundos. A coluna na esquerda (1, 2, 4, 8, 16) indica o número usado para~\code{-n}. A linha superior indica a qual das versões do trabalho pertence a medição. No caso da primeira versão do segundo problema, o número de sub-intervalos (32, 128, 512, 1024) está indicado entre parenteses.

\begin{table}[h]
\begin{tabular}{|c|c|c|c|c|c|c|}
\hline
    { }                     &
    {\textbf{P1}}           &
    {\textbf{P2-V1 (32)}}   &
    {\textbf{P2-V1 (128)}}  &
    {\textbf{P2-V1 (512)}}  &
    {\textbf{P2-V1 (1024)}} &
    {\textbf{P2-V2}}        \\
\hline
{\textbf{1}}  & 4.628   & -     & -     & -     & -     & -       \\ \hline
{\textbf{2}}  & 3.376   & 5.116	& 5.295	& 5.311	& 4.576	& 35.040  \\ \hline
{\textbf{4}}  & 2.626   & 1.825	& 1.631	& 1.877	& 1.840	& 14.191  \\ \hline
{\textbf{8}}  & 2.025   & 1.166	& 0.860	& 0.770	& 0.803	& 59.562  \\ \hline
{\textbf{16}} & -       & 1.028	& 0.697	& 0.532	& 0.468	& 154.004 \\ \hline
\end{tabular}
\end{table}

A execução mais rápida foi o \textbf{P2-V1 (1024)}, terminando em 0.468 segundos, que é a que mais divide o intervalo à ser calculado, tornando os trabalhos de cada nó mais fáceis. Vemos que os piores tempos de execução são do programa utilizando a solução do problema 2 versão 2, onde há mais troca de mensagens entre os nós.

\section*{Desenvolvimento}

O código do trabalho está dividido entre três diferentes arquivos \code{main.c}, contendo as diferentes versões do programa.

\begin{itemize}
\item Para \textbf{P1}, o programa faz apenas um \code{MPI\_Reduce} para sincronizar os dados e calcular a área total. Sua implementação é bem simples.

\item Para \textbf{P2-V1}, nós mantemos no nó mestre uma lista de intervalos a serem calculados, que é esvaziada conforme nós trabalhadores requisitam novos intervalos para calcular. O programa termina quando o contador \code{splits}, que supervisiona a quantidade de intervalos calculados, chega a zero.

\item Para \textbf{P2-V2}, a lógica é similar à da versão anterior, com a diferença da existência de um novo tipo de mensagem \code{NEW\_INTERVAL}, que incrementa a variável \code{splits} e adiciona um novo intervalo na lista do nó mestre. Essa mensagem é enviada por um nó trabalhador quando ele divide um intervalo em dois.
\end{itemize}

O código se encontra em: \textit{https://github.com/mambrozio/pcp-assignment-02}

\end{document}
